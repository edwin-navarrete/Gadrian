Welcome to the user manual and scripting reference for Grid Framework. This document consists of two parts, the manual and the A\+P\+I reference. Read the manual to get a good understanding of what Grid Framework is and how it works, and read the reference for details about the A\+P\+I. Aside from that you can also jump straight into the included examples (in your project view under {\itshape Grid Framework/\+Examples}) and see Grid Framework right in action.

An online version of this documentation can be found under \href{http://hiphish.github.io/grid-framework/documentation/}{\tt this U\+R\+L}. Feel free to bookmark the link and and throw away the local documentation files. They can be found in the {\itshape Web\+Player\+Templates} folder of your project. If you delete the local documentation the help menu will automatically forward you to the online U\+R\+L as well.

\begin{DoxyNote}{Note}
Some images have been scaled down automatically in order to maintain the flow of text. In that case you have to right-\/click them and choose to open in a new tab or window to see the image in its full resolution. The maximum size of images scales with font size, so increasing font size will enlarge the images as well until they reach their native size.
\end{DoxyNote}
\subsection*{Table of Contents }

The user manual consist of the following pages\+:
\begin{DoxyItemize}
\item Page \hyperlink{overview}{Overview} gives you a rundown of all included files.
\item Page \hyperlink{concept_of_grid}{Understanding the concept of grids} explains the general idea behind grids in Grid Framework.
\item Page \hyperlink{abstract_grids}{Abstract Grids} contains information on the bases class all grids inherit from.
\item Page \hyperlink{rectangular_grid}{Rectangular Grids} discusses rectangular grids.
\item Page \hyperlink{hex_grid}{Hexagonal grids} discusses hexagonal grids.
\item Page \hyperlink{polar_grid}{Polar grids} discusses polar grids.
\item Page \hyperlink{rendering_drawing}{Rendering and drawing a Grid} explains to display grids in the editor and in the game and the associated performance costs.
\item Page \hyperlink{getting_started}{Getting Started} explains how to find the features you are looking for in the interface.
\item Page \hyperlink{events}{Events} features an introductory example for built-\/in events in Grid Framework.
\item Page \hyperlink{extending}{Extending Grid Framework} demonstrates two ways of extending the capabilities of Grid Framework to your liking without changing the source code.
\item Page \hyperlink{debugging}{Debugging a grid} explains hot to debug and analyze your grids.
\item Page \hyperlink{legacy_support}{Legacy Support} contains information on features that were either dropped or changed and how to adapt your code.
\item Page \hyperlink{playmaker_support}{Playmaker support} discusses the built-\/in Playmaker actions and how to write your own actions. (Playmaker is a separate addon for Unity and not affiliated with Grid Framework)
\item Page \hyperlink{changelog}{Changelog} contains the changelog, it's the same file you can also find under Grid Framework/\+Changelog.
\end{DoxyItemize}

You can find the A\+P\+I scripting reference under {\itshape Classes} either in the sidebar or in the top bar. Form there you have various sorting options.

\begin{DoxyNote}{Note}
Due to limitations of Doxygen the A\+P\+I can only be documented with C\# syntax. Despite this you can still write your code in Unity\+Script, there are no C\# exclusive features. Most of the examples are written in both languages as well, except for the ones that make use of special language features not present in Unity\+Script. 
\end{DoxyNote}
